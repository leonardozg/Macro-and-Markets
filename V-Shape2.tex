% Options for packages loaded elsewhere
\PassOptionsToPackage{unicode}{hyperref}
\PassOptionsToPackage{hyphens}{url}
%
\documentclass[
]{article}
\usepackage{lmodern}
\usepackage{amssymb,amsmath}
\usepackage{ifxetex,ifluatex}
\ifnum 0\ifxetex 1\fi\ifluatex 1\fi=0 % if pdftex
  \usepackage[T1]{fontenc}
  \usepackage[utf8]{inputenc}
  \usepackage{textcomp} % provide euro and other symbols
\else % if luatex or xetex
  \usepackage{unicode-math}
  \defaultfontfeatures{Scale=MatchLowercase}
  \defaultfontfeatures[\rmfamily]{Ligatures=TeX,Scale=1}
\fi
% Use upquote if available, for straight quotes in verbatim environments
\IfFileExists{upquote.sty}{\usepackage{upquote}}{}
\IfFileExists{microtype.sty}{% use microtype if available
  \usepackage[]{microtype}
  \UseMicrotypeSet[protrusion]{basicmath} % disable protrusion for tt fonts
}{}
\makeatletter
\@ifundefined{KOMAClassName}{% if non-KOMA class
  \IfFileExists{parskip.sty}{%
    \usepackage{parskip}
  }{% else
    \setlength{\parindent}{0pt}
    \setlength{\parskip}{6pt plus 2pt minus 1pt}}
}{% if KOMA class
  \KOMAoptions{parskip=half}}
\makeatother
\usepackage{xcolor}
\IfFileExists{xurl.sty}{\usepackage{xurl}}{} % add URL line breaks if available
\IfFileExists{bookmark.sty}{\usepackage{bookmark}}{\usepackage{hyperref}}
\hypersetup{
  pdftitle={Recent Financial Cycles- Statistical Features and Parameters},
  pdfauthor={Leonardo Zepeda},
  hidelinks,
  pdfcreator={LaTeX via pandoc}}
\urlstyle{same} % disable monospaced font for URLs
\usepackage[margin=1in]{geometry}
\usepackage{longtable,booktabs}
% Correct order of tables after \paragraph or \subparagraph
\usepackage{etoolbox}
\makeatletter
\patchcmd\longtable{\par}{\if@noskipsec\mbox{}\fi\par}{}{}
\makeatother
% Allow footnotes in longtable head/foot
\IfFileExists{footnotehyper.sty}{\usepackage{footnotehyper}}{\usepackage{footnote}}
\makesavenoteenv{longtable}
\usepackage{graphicx,grffile}
\makeatletter
\def\maxwidth{\ifdim\Gin@nat@width>\linewidth\linewidth\else\Gin@nat@width\fi}
\def\maxheight{\ifdim\Gin@nat@height>\textheight\textheight\else\Gin@nat@height\fi}
\makeatother
% Scale images if necessary, so that they will not overflow the page
% margins by default, and it is still possible to overwrite the defaults
% using explicit options in \includegraphics[width, height, ...]{}
\setkeys{Gin}{width=\maxwidth,height=\maxheight,keepaspectratio}
% Set default figure placement to htbp
\makeatletter
\def\fps@figure{htbp}
\makeatother
\setlength{\emergencystretch}{3em} % prevent overfull lines
\providecommand{\tightlist}{%
  \setlength{\itemsep}{0pt}\setlength{\parskip}{0pt}}
\setcounter{secnumdepth}{-\maxdimen} % remove section numbering

\title{Recent Financial Cycles- Statistical Features and Parameters}
\author{Leonardo Zepeda}
\date{01/02/2021}

\begin{document}
\maketitle

\hypertarget{abstract}{%
\subsection{Abstract}\label{abstract}}

By analyzing DOW and S\&P indexes from 1980 to date, the author
identifies 7 financial cycles and present common statistical features
and parameters for each cycle. The analysis found that a cycle's length
range from 764 to 1,777 days, averaging 20 basis points (bp) and 110 bp
weekly and monthly returns respectively for DOW index; and 21 bp and 114
bp weekly and monthly returns respectively for S\&P index. The analysis
also found that cycle's total returns averaged 1.58 and 1.59 times it
initial values for DOW and S\&P indexes respectively, ranging from 1.38
to 1.81 times for DOW and from 1.38 to 1.81 times for S\&P. With the
latest, and considering only statistical features, we could infer
current cycle have between 531 to 1544 days left, with at least 11\% of
those days with negative returns, but with an expectation of delivering
at least 1.58 times its initial values.

\hypertarget{sources-of-data}{%
\subsection{Sources of Data}\label{sources-of-data}}

The data comes from Yahoo Finance, selecting the adjusted price column.
These data correct the irregularities generated by idiosyncratic events
of the firms that are not market movements (dividends, splits, etc.).
Thus, the data are comparable over time. From them we create new
columns: returns (daily percentage change), lagged price and lagged
performance. Lagging performance is simply the performance of the
previous day.

\hypertarget{load-and-transform-data}{%
\subsection{Load and Transform Data}\label{load-and-transform-data}}

Our objective is to find out critical statistical features to describe a
cycle: length, weekly returns, monthly returns, total returns per cycle,
and other technical (statistical) features and parameters useful for
risk management.

\hypertarget{a-first-visual-analysis}{%
\subsection{A First Visual Analysis}\label{a-first-visual-analysis}}

By observing historical levels we notice dramatic changes against
certain dramatic events. More relevant and dramatic are 1987 crisis,
9/11 in 2001 and the sub-prime crisis in 2009 and more recently the
dramatic COVID-19 crisis in 2020. Is also noticeable that indexes behave
highly correlated over time.

\hypertarget{graph-i--levels}{%
\subsubsection{Graph I -Levels}\label{graph-i--levels}}

\includegraphics{V-Shape2_files/figure-latex/unnamed-chunk-4-1.pdf}

\hypertarget{leveling-perspective}{%
\subsection{Leveling Perspective}\label{leveling-perspective}}

Price's magnitude distract the observer from the size of the variations.
By considering price levels of 2020 (around 30,000 points for DOW) the
observer may perceive a reduced importance of price variations in say,
1987 (when prices of DOW were one-tenth of 2020 levels). By using
returns we can focus in the magnitude of the change avoiding the
distraction of increasing price levels. In order to filter only
``sustained'' variations, used returns are on weekly and monthly basis

\hypertarget{graph-ii--weekly-returns}{%
\subsubsection{Graph II -Weekly
Returns}\label{graph-ii--weekly-returns}}

\includegraphics{V-Shape2_files/figure-latex/unnamed-chunk-6-1.pdf}

\hypertarget{graph-iii--monthly-returns}{%
\subsubsection{Graph III -Monthly
Returns}\label{graph-iii--monthly-returns}}

\includegraphics{V-Shape2_files/figure-latex/unnamed-chunk-7-1.pdf}
\#\#\# Is there a trend for recovery ?

One of the interesting aspects to analyze is to find out if there is a
noticeable trend for recovery after sensitive falls. First we will
analyze global patterns of recovery by plotting returns against lagged
returns in weekly (7 days) and monthly basis (30 days). If immediate or
abrupt recovery was a pattern, we should expect a more dense cloud of
points in the higher-left corner of the graph, meaning that sensitive
falls corresponds to sensitive recovery. As observed in following graph
neither for DJI or S\&P 500 and weekly nor monthly data suggest that it
would be the case.

\hypertarget{graph-iv--recovery-paterns}{%
\subsubsection{Graph IV -Recovery
Paterns}\label{graph-iv--recovery-paterns}}

\includegraphics{V-Shape2_files/figure-latex/unnamed-chunk-8-1.pdf}

To validate the latest inference, we arrange data in ``equally-sized
volatility clusters'' regardless of the time of occurrence. We arrange
data in 100 equally sized groups. We call them Similar Volatility
Groups. As a result from that arrangement, the first group will contain
1\% of the smallest lagged returns, while the latest group will contain
the 1\% largest lagged returns. Then for each group we calculate its
mean and the positive portion of each group.

By filtering only current positive returns, we observe that the groups
with the smallest returns correspond to negative mean lagged returns and
the largest returns correspond to the largest positive lagged mean
returns in asyntotical line around the 0 mean. The latest complies for
both weekly and monthly lagged returns. Up to now it might be confirmed
that there is not a noticeable trend for recovery after sensitive falls.

Moreover, when plotting its density we find that there is sensitively
more density of positive cases, within the 50\% larger lagged returns.
This may imply that the magnitude of the recovery is more often related
with previous positive returns, than with sensitive falls. A momentum
argument as commonly referred by traders.

We will come later again with this global result of the data.

\begin{verbatim}
## `summarise()` regrouping output by 'index' (override with `.groups` argument)
## `summarise()` regrouping output by 'index' (override with `.groups` argument)
\end{verbatim}

\hypertarget{graph-v--recovery-paterns-in-depth}{%
\subsection{Graph V -Recovery Paterns in
depth}\label{graph-v--recovery-paterns-in-depth}}

\includegraphics{V-Shape2_files/figure-latex/unnamed-chunk-10-1.pdf}

\hypertarget{features-of-cyles}{%
\subsection{Features of cyles}\label{features-of-cyles}}

There are several methods and criteria to identify cycles. Far from any
theoretical stance and with a mere practical objective, in this analysis
we define the start of a cycle with a threshold of a 10\% or bigger
daily fall, coincident with a 10\% weekly and a 10\% monthly fall in
index prices. As arbitrary as it seems, it provides 7 coincident falls
(for both indexes) from 1985 to 2021 (see Graph 2 and 3)

\hypertarget{critical-dates}{%
\subsubsection{Critical dates}\label{critical-dates}}

\begin{longtable}[]{@{}ll@{}}
\toprule
Date & Index\tabularnewline
\midrule
\endhead
1987/10/19 & DOW, S\&P\tabularnewline
1998/08/31 & DOW, S\&P\tabularnewline
2001/09/17 & DOW, S\&P\tabularnewline
2002/07/22 & S\&P\tabularnewline
2008/10/07 & DOW, S\&P\tabularnewline
2009/02/23 & S\&P\tabularnewline
2011/08/08 & DOW, S\&P\tabularnewline
2015/08/25 & DOW, S\&P\tabularnewline
2020/02/27 & DOW, S\&P\tabularnewline
---------- & ----------\tabularnewline
\bottomrule
\end{longtable}

To identify a cycle we consider only critical dates coincident for both
indexes what excludes: 2002/07/22 and 2009/02/23 from the S\&P index.

\hypertarget{length-of-cycles}{%
\subsubsection{Length of Cycles}\label{length-of-cycles}}

Cycle length is calculated by the number of days between one 10\%+
(daily, weekly and monthly) coincident fall and the next one in
occurrence. Length cycle is represented in Graph VI. Please note that
since the start date of series is arbitrary, cycle one for both index
are not consistent. Also please note that cycle 7 is at early stage.

Considering the latest we have 5 cycles to analyze. Graph 4 includes an
horizontal line in 1081 days, representing the mean of cycles length of
the 5 analyzed (complete) cycles.

\hypertarget{graph-vi--length-of-cycles}{%
\subsubsection{Graph VI -Length of
Cycles}\label{graph-vi--length-of-cycles}}

\includegraphics{V-Shape2_files/figure-latex/unnamed-chunk-13-1.pdf}

\hypertarget{potential-return-of-cycles}{%
\subsubsection{Potential Return of
Cycles}\label{potential-return-of-cycles}}

Maximum Return of each cycle is calculated with the highest normalized
return reached by each index per each cycle. As noted in graph VII
cycles are slightly above its average of 1.583 times its initial level
being cycles 5 and 6 the highest with 1.63 and 1.81 for DOW and 1.82 and
1.74 for S\&P.Current cycle stands already in 1.24 for Dow and 1.33 for
S\&P. So despite is still quite short in time is has already run most of
its potential return.

\hypertarget{graph-vii--potential-return-of-cycles}{%
\subsubsection{Graph VII -Potential Return of
Cycles}\label{graph-vii--potential-return-of-cycles}}

\includegraphics{V-Shape2_files/figure-latex/unnamed-chunk-14-1.pdf}

\hypertarget{graph-vii--cycles}{%
\subsubsection{Graph VII -Cycles}\label{graph-vii--cycles}}

\includegraphics{V-Shape2_files/figure-latex/unnamed-chunk-15-1.pdf}

\hypertarget{weekly-returns--descriptive-statistics-per-cycle}{%
\subsubsection{Weekly Returns -Descriptive statistics per
cycle}\label{weekly-returns--descriptive-statistics-per-cycle}}

Besides length of the cycle, central tendency analysis may reveal
another important features. Every cycle present different weekly
features regarding mean returns, volatility (standard deviation),
fewness, and qurtosis. All this features are represented in the
following graph and tables. We will return to this features later (when
it become useful for inference and interpretation).

\hypertarget{graph-viii--weekly-returns--distribution-of-returns}{%
\subsubsection{Graph VIII -Weekly Returns -Distribution of
returns}\label{graph-viii--weekly-returns--distribution-of-returns}}

\includegraphics{V-Shape2_files/figure-latex/unnamed-chunk-16-1.pdf}

\begin{longtable}[]{@{}rrrrr@{}}
\caption{DOW Cycles Weekly statistical features}\tabularnewline
\toprule
days & mean(bp) & sd & skewness & kurosis\tabularnewline
\midrule
\endfirsthead
\toprule
days & mean(bp) & sd & skewness & kurosis\tabularnewline
\midrule
\endhead
1648 & 29.41352 & 0.0173122 & -0.6107945 & 2.8320741\tabularnewline
764 & 14.14445 & 0.0268605 & -0.0701918 & 0.6651342\tabularnewline
1777 & 4.29654 & 0.0226467 & -0.3338475 & 4.7871631\tabularnewline
714 & 13.55298 & 0.0323004 & -0.6031292 & 5.5733275\tabularnewline
1018 & 19.49517 & 0.0187700 & -0.4516548 & 3.2169428\tabularnewline
1134 & 24.12576 & 0.0185049 & -1.0013622 & 4.0584951\tabularnewline
249 & 38.76379 & 0.0442727 & -0.7258756 & 5.3324439\tabularnewline
\bottomrule
\end{longtable}

\begin{longtable}[]{@{}rrrrr@{}}
\caption{S\&P Cycles Weekly statistical features}\tabularnewline
\toprule
days & mean(bp) & sd & skewness & kurosis\tabularnewline
\midrule
\endfirsthead
\toprule
days & mean(bp) & sd & skewness & kurosis\tabularnewline
\midrule
\endhead
4682 & 25.836978 & 0.0214184 & -1.0254654 & 12.5178094\tabularnewline
764 & 7.196581 & 0.0285053 & -0.0683870 & 0.2822286\tabularnewline
1777 & 2.434895 & 0.0227557 & -0.3101766 & 4.1413968\tabularnewline
714 & 13.845379 & 0.0355070 & -0.5846155 & 5.0250587\tabularnewline
1018 & 26.095323 & 0.0199758 & -0.4669984 & 3.6606779\tabularnewline
1134 & 22.513281 & 0.0181972 & -1.0879144 & 4.2767242\tabularnewline
249 & 50.838146 & 0.0398337 & -0.8545240 & 5.0465974\tabularnewline
\bottomrule
\end{longtable}

\hypertarget{monthly-returns}{%
\subsubsection{Monthly Returns}\label{monthly-returns}}

\hypertarget{monthly-returns--descriptive-statistics-per-cycle}{%
\subsubsection{Monthly Returns -Descriptive statistics per
cycle,}\label{monthly-returns--descriptive-statistics-per-cycle}}

Besides length of the cycle, central tendency analysis may reveal
another important features. Every cycle present different monthly
features regarding mean returns, its volatility (standard deviation),
its fewness, and kurtosis. All this features are represented in the
following graph and tables. We will return to this features later (when
it become useful for inference and interpretation).

\hypertarget{graph-ix--distribution-of-returns}{%
\subsubsection{Graph IX -Distribution of
returns}\label{graph-ix--distribution-of-returns}}

\includegraphics{V-Shape2_files/figure-latex/unnamed-chunk-18-1.pdf}

\begin{longtable}[]{@{}rrrrr@{}}
\caption{DOW Cycles Monthly statistical features}\tabularnewline
\toprule
days & mean(bp) & sd & skewness & kurosis\tabularnewline
\midrule
\endfirsthead
\toprule
days & mean(bp) & sd & skewness & kurosis\tabularnewline
\midrule
\endhead
1648 & 186.40461 & 0.0374510 & -0.1262172 & 0.7131017\tabularnewline
764 & 84.43536 & 0.0618898 & 0.0411795 & 0.0298988\tabularnewline
1777 & 25.95773 & 0.0500465 & -0.7075157 & 1.3842516\tabularnewline
714 & 75.00987 & 0.0741046 & -0.9942326 & 1.8947280\tabularnewline
1018 & 110.66126 & 0.0361364 & -0.6245191 & 1.7710250\tabularnewline
1134 & 142.16087 & 0.0423092 & -0.4828789 & 0.6535559\tabularnewline
249 & 151.96202 & 0.1031907 & -1.5004530 & 2.8573753\tabularnewline
\bottomrule
\end{longtable}

\begin{longtable}[]{@{}rrrrr@{}}
\caption{S\&P Cycles Monthly statistical features}\tabularnewline
\toprule
days & mean(bp) & sd & skewness & kurosis\tabularnewline
\midrule
\endfirsthead
\toprule
days & mean(bp) & sd & skewness & kurosis\tabularnewline
\midrule
\endhead
4682 & 158.33327 & 0.0496389 & -0.7231017 & 3.9134350\tabularnewline
764 & 42.83200 & 0.0615426 & 0.0827184 & 0.1679607\tabularnewline
1777 & 18.64870 & 0.0503776 & -0.7763416 & 1.4838870\tabularnewline
714 & 68.92202 & 0.0822561 & -0.9564228 & 2.1679159\tabularnewline
1018 & 146.53337 & 0.0385787 & -0.8213659 & 2.1379069\tabularnewline
1134 & 130.80032 & 0.0402066 & -0.6520906 & 0.9764067\tabularnewline
249 & 242.78640 & 0.0982475 & -1.4785692 & 2.7777321\tabularnewline
\bottomrule
\end{longtable}

\includegraphics{V-Shape2_files/figure-latex/unnamed-chunk-20-1.pdf}
\includegraphics{V-Shape2_files/figure-latex/unnamed-chunk-20-2.pdf}

Achieve \textless- tapply(df\(norm, list(df\)index, df\$cycle), FUN=max)
Achieve \textless- Achieve{[}, c(2:6){]}

\end{document}
